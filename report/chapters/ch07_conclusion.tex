%!TEX root = ../report.tex
\documentclass[report.tex]{subfiles}
\begin{document}
\chapter{Conclusion}
In this research and development project, the possibility of exploiting contact constraint in robotic manipulation task has been presented under the experiments with three use cases: grasping object by sliding along a surface, performing writing task and resting robot manipulation.

At the beginning of the report, the state of the art has been presented to introduce the basic concept related to this research and development project. By delivering the concept of dynamics, the related solver: The Articulated Body Algorithm (ABA), Recursive Newton-Euler Algorithm (RNEA) and Articulated-Body Hybrid Dynamics Algorithm (ACHDA) ,we pinpoint out the deficit of the algorithm mentioned above. According to the algorithm delivered in \ref{alg:ABA} and \ref{alg:ALG1}, they all require a full task specification such as all 6 DoF of joint acceleration or joint torque to compute the forward or reverse dynamic which lacks flexibility in real life implementation. Also the concept of controller and some common examples are presented such as Proportional-integral-derivative (PID) controllers, Fuzzy logic algorithm and Impedance controllers. To keep the implementation simple, a cascaded controller with two P-controller is being used. After that the concept of Task based control is delivered to give a brief introduction on how to run a motion by specifying a task instead of directly control individual joints or degree of freedom.

The deficit of state of the art leads to the introduction of Popov-Vereshchagin hybrid solver, and related library being used in this project in chapter \ref{Background}. The description of the solver, that is the input and the resulting output, have be discussed. Then the task interfaces of Popov-Vereshchagin hybrid solver is being introduced. There are three task interfaces can be specified as input of the solver: The external force $F_{ext}$, the feed-forward force $\tau$ and Cartesian acceleration constraints $\alpha_N^T \ddot{X_N} = \beta_N$. The latest one is the core of the solver where we can assign partial constraint in some degrees of freedom (DOF) instead of a full six DOFs, for example disabling the constraint on linear z direction such that the motion in this direction is defined by nature (gravity). This partial constraint stands out compare to the common dynamic solvers mentioned above. In KDL library, there is an existing function that implemented the Popov-Vereshchagin hybrid solver. In coding, KDL present a overloading function KDL::diff() to calculate the difference between frames, twist .etc which is very practical in implementation stage as we discussed in \ref{diff_frame}.

With the background of Popov-Vereshchagin (PV) hybrid solver and related KDL library are being introduced. Chapter \ref{Solution} presented the proposed robot architecture of this research and development project. Their system mainly consists of the robot, the PV hybrid solver, and a cascaded controller with the KDL Parser library to transform the urdf file of the robot manipulator to a KDL chain for dynamic computation. Besides the robot architecture, in software development, instead of using the existing Kinova Kortex API, a robot control interface Robif2b is being used. As we discussed in \ref{Robif2b}, the control interface was lacking connection to the actuator voltage and gripper control. This led to an extension of Robif2b such that we can achieve the function aforementioned.

After discussing the software part of the development, three experiments were conducted according to the use cases. There are three use cases in this project: Grasping objects by sliding motion along the surface, performing writing tasks, and resting elbow manipulation. The experiment designs, setups, and evaluations are presented according to use cases. In use case 1, the general result is expected where the average power consumption of each joint except joint 3 decreases in the contact case, which matches the hypothesis but with an unexpected result where a certain joint has higher energy consumption due to overcoming the friction with the contact surface. In the use case 2, it is surprising that the trajectory of the contact case is less accurate as presented in \ref{fig:us2_traj_com_plot} since from a human perspective it is more stable to have our wrist on the contact surface. This may be due to the friction of the surface. The last use case is to study the energy efficiency of resting other joints, for example, an elbow joint on a surface, and perform a manipulation task. According to use case 1, we believe that energy efficiency may increase in such a situation. Additionally, a future investigation can focus on different contact surfaces in terms of texture and shape. With the surprising result in \ref{us1}, it is possible to study how different tasks affect the energy consumption of the robot manipulator because the motion is diver between different tasks and it is possible that other joints would experience a higher torque to overcome the friction just like joint 3 in use case1. From the observation in \ref{us2} and \ref{us3}, it is possible to study multiple joint resting cases to see if the accuracy increases when the robot task is executed under such a setup.

Future work on the topic of exploiting contact constraints in robotic manipulation tasks could look into fusing force/torque sensor on the robot manipulator since the Kinova arms does not have F/T sensors on the gripper so the task execution is lacking feedback from the gripper to acknowledge the system a gripping task is finished.
\end{document}
