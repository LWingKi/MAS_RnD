%!TEX root = ../report.tex
\documentclass[report.tex]{subfiles}
\begin{document}
    \chapter{Problem Statement}
    Many of today's robots are characterized by their substantial weight, high energy consumption, and expensive manufacturing costs. In the context of motion planning, contact surfaces are frequently regarded as hindrances, resulting in the robot's joints being suspended in mid-air. This approach can lead to wasteful energy usage as the robot expends energy to maintain these joints and links in such a suspended state. Furthermore, current dynamic solvers like the Articulated Body Algorithm (ABA) and the Recursive Newton-Euler Algorithm (RNEA) are ill-suited for handling partial constraint conditions. The Popov-Vereshchagin hybrid solver, although capable of accommodating such task specifications, remains relatively unknown within the robotics community, with a lack of comprehensive studies and implementations.

    As humans, we go about our daily tasks by resting our arms on surfaces. Taking inspiration from this common human behaviour and the unique characteristics of the Popov-Vereshchagin hybrid solver, it becomes conceivable that robots could replicate such motions.
    
    In the scope of our research and development project, we intend to explore an innovative approach that addresses the dynamics of this scenario. By re-framing contact surfaces as opportunities rather than obstacles, our objective is to cultivate a robotic motion planning solution that is not only more energy-efficient but also more cost-effective.
\end{document}