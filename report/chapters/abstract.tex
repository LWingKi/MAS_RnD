%!TEX root = ../report.tex
\documentclass[report.tex]{subfiles}
\begin{document}
    \begin{abstract}
        During the process of motion planning, robot manipulators often view contact surfaces as obstacles rather than opportunities. This approach stands in contrast to human behavior, where people frequently rest their wrists on a table while writing on paper. To enable robots to emulate this behavior, there exists a dynamic solver capable of addressing such requirements. The Popov-Vereshchagin hybrid solver allows for partial motion specifications in certain directions, allowing them to be influenced by natural forces. However, this solver remains relatively obscure within both academic and industrial circles, and there is limited research and implementation related to this topic.

        The objective of the research and development project is to investigate the impact of robots leveraging environmental constraints. Three experiments were conducted to assess the energy efficiency and motion accuracy of robot manipulation when assigned to everyday tasks such as writing.

        In the evaluation, it was observed that the average energy consumption and joint torque of most of the robot's joints decreased when performing a grasping task. Interestingly, the accuracy of motion in contact scenarios was found to be lower than in contactless situations. This discrepancy may be attributed to the need to overcome friction with the contact surface, thereby opening up opportunities for further research into the exploitation of contact constraints across different textures and shapes of contact surfaces. Additionally, it explores how different tasks affect the energy efficiency of leveraging contact surfaces during robot manipulation tasks.
    \end{abstract}
\end{document}
