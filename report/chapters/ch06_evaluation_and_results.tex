%!TEX root = ../report.tex
\documentclass[report.tex]{subfiles}
\begin{document}
\chapter{Evaluation and results}
In this RnD, two experiements are being conducted with respected to two use cases: (1) Grasp object by sliding motion along surface and (2) Perform writing task. In this chapter, the methodology will be divided according to use cases.
    \section{Use case 1 - Grasp object by sliding motion along surface} 
    The aim of this case study is to grasp the object successfully by sliding the robot
    manipulator along a contact surface. Assume that the object pose is known, 
    only the object to be grasped is on the table , no obstacle  and the contact surface is known.
    The manipulator should first approach the contact surface, for example, a table
    with the target object placed on it. When the manipulator is above the contact
    surface, it moves toward the surface until establishing contact. By activly monitoring the velocity along linear Z axis $v_{lin_z}$in world frame.
    If the absolute value of $v_{lin_z}$ for 10 samples is less than a threshold value. The contact between a surface and the robot manipulator is being established.
    After contact is established, the manipulator slides along the linear x direction for 10 cm $d_{x} = 0.1$ until it reaches a grasping
    region. Finally, the end-effector performs a grasping motion.
        \subsection{Setup}
        \begin{itemize}
            \item Kinova® Gen3 Ultra lightweight robot maniputlator is attached to a table
            \item The motion will starts at a fixed starting pose $q = {6.28318,0.261895,3.14159,4.01417,0,0.959856,1.57079}$ in radian
        \end{itemize}
        \subsection{Experimental Design}
            \begin{itemize}
                \item 
            \end{itemize}

    \section{Use case 2 - Perform writing task} 
    The aim is to draw a line on the paper with a wrist joint contacting the writing
    surface like human writing. Before starting the manipulation task, the gripper
    firmly grasps the pen or marker. The manipulator should first approach the
    contact surface, for example, a table with the target object placed on it. When
    the manipulator is above the contact surface, it moves toward the surface until
    establishing contact. Once contact is established, the manipulator draws a line
    according to a predefined motion specification. The evaluation will be a trajectory
    comparison in terms of position or velocity with and without contact between the
    robot and the support surface
        \subsection{Setup}
        \subsection{Experimental Design}
        \begin{itemize}
            \item 
        \end{itemize}
    \section{Use case 3 - Resting elbow manipulation}
        \subsection{Experimental Design}
        \begin{itemize}
            \item 
        \end{itemize} 


%%%%
    % \chapter{Evaluation and Results}

    % \section{Experiment Description}

    % Describe the experiments/evaluation you are performing to analyse your method.

    % \section{Experimental Setup}  

    % Describe your experimental setup in detail.

    % \section{Results}

    % Describe the results of your experiments in detail.

\end{document}
